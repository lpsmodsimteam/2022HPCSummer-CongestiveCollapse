\documentclass{article}

%graphics
\usepackage{graphicx}
\graphicspath{{./images/}}

\usepackage{float}

% margins of 1 inch:
\setlength{\topmargin}{-.5in}
\setlength{\textheight}{9.5in}
\setlength{\oddsidemargin}{0in}
\setlength{\textwidth}{6.5in}

\usepackage{hyperref}
\hypersetup{
    colorlinks=true,
    linkcolor=blue,
    filecolor=magenta,      
    urlcolor=cyan,
    pdftitle={Overleaf Example},
    pdfpagemode=FullScreen,
    }

\begin{document}

    % https://stackoverflow.com/a/3408428/1164295
    \begin{minipage}[h]{\textwidth}
        \title{2022 Future Computing Summer Internship Project:\\(Modeling congestive collapse in a discrete event simulator to find metrics that measure its existence.)}
        \author{Nicholas Schantz\footnote{nickjohnschantz@gmail.com}\ , 
        anotherfirst\footnote{anemail@domain.com}}
        \date{\today}
            \maketitle
        \begin{abstract}
            Congestive collapse is an issue involving reliable network protocols that send large amounts of retransmission clogging a link with duplicate data. This leads the network in a state where useful throughput has vastly decreased or does not exist over time. SST will be used to model the problem and parameterized so that the problem can be simulated and metrics can be collected which identify if congestive collapse exist. We determine that the ratio of goodput and throughput in the link can assist in identifying the existing of the problem. Furthermore, we determine that a link's send queue depth and ratio of first packets to duplicate packets directly measure the existence of congestive collapse.
        \end{abstract}
    \end{minipage}

\ \\
% see https://en.wikipedia.org/wiki/George_H._Heilmeier#Heilmeier's_Catechism

%\maketitle

\section{Project Description} % what problem is being addressed? 

The challenge addressed by this work is to map the networking problem congestive collapse to a discrete event simulator. Sandia National Laboratories Structural Simulation Toolkit (SST) is a discrete event simulator framework which is used to simulate the problem. The problem is simulated to identify mathematical conditions that cause the problem. This information is used to develop metrics to identify that the problem has occurred in generalized systems.

\section{Motivation} % Why does this work matter? Who cares? If you're successful, what difference does it make?

Identifying the metrics for detecting congestive collapse will be vital for developing distributed systems that can avoid congestive collapse from occurring during communication. Furthermore, the metrics answer the question as to why the problem has occurred in the system. To add, the SST models written are resources that other users can use to learn and utilize SST's discrete event simulator.

\section{Prior work} % what does this build on?


\section{How to do the thing}

The software developed to respond to this challenge was run on one laptop.
The software is available on (https://github.com/lpsmodsim/2022HPCSummer-CongestiveCollapse)

\section{The Model}

There are two models that simulate the behavior of congestive collapse.
Both models contain a \textit{n} sender node that send data to a single receiver node.

\begin{figure}[H]
	\includegraphics[scale=0.5]{model}
	\centering
	\caption{Example of model with one sender and receiver node}
\end{figure}

The models differ by the properties of the sender and receiver nodes.
In the first model, the sender will send a set number of packets per tick while the receiver will process a set number of packets in its queue.
In the second model, the sender and receiver both send and process packets one time a cycle. This was done to better implement the sliding window sending process for the TCP reliable networking protocol.
Both model's receiver nodes will have an infinite queue to avoid implementing packet loss.

\section{Result} % conclusion/summary

To determine if congestive collapse exist in a network. Three potential metrics were determined:\newline
\indent Calculating a ratio between the useful throughput and the total throughput of a link gives insight to how many duplicate or corrupted packets have passed through a link. However, if this ratio is low due to corrupted packets, that does not indicate congestive collapse. In this case queue depth is measured and if it is largely being filled with duplicate packets rather than new packets, then we can declare that the network is being flooded with retransmissions. In this case these metrics can identify that congestive collapse exist in the network.\newline
\indent An example of the problem is simulated and gathered data is shown below. In this SST simulation, a node is sending packets at 10 times the rate that a receiver node can process them at. In this case the receiver node's queue fills quickly and the receiver is unable to send acknowledgments under the sender's timeout time. Retransmissions are sent which cause more packets to fill the queue and the useful throughput exponentially decays to zero in a short span of time. \newline\newline

\begin{figure}[H]
\includegraphics[scale=0.5]{throughput}
\centering
\caption{Ratio of goodput to throughput over a span of 100 seconds}
\end{figure}

\begin{figure}[H]
	\includegraphics[scale=0.5]{newpackets}
	\centering
	\caption{Amount of new packets entering the queue each second for 100 second}
\end{figure}

\begin{figure}[H]
	\includegraphics[scale=0.5]{queuedepth}
	\centering
	\caption{Infinite queue which increases over time.}
\end{figure}

\section{Future Work}



\end{document}
